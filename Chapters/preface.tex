\chapter*{Prefácio}
\addcontentsline{toc}{chapter}{Prefácio}
Este material foi escrito como supórte para a disciplina SEL609 da Universidade de São Paulo, Escola de Engenharia de São Carlos.

\section*{Ementa}
A disciplina abordará os seguintes dispositivos:
\begin{itemize}
  \item Amplificadores Operacionais
  \item Diodos 
  \item Transistores Bipolares 
\end{itemize}

Além disso técnicas de análise de circuitos serão abordadas e/ou revisitadas:
\begin{itemize}
  \item Análise de malhas e nós 
  \item Transformada de Laplace
  \item Diagramas de Bode
\end{itemize}

E configurações de circuitos básicos:
\begin{itemize}
  \item Amplificadores Operacionais 
    \begin{itemize}
      \item Circuito de ganho
      \item Circuitos derivados 
      \item Circuitos integrados
    \end{itemize}
  \item Diodo (Incluindo Zener e Shottky)
    \begin{itemize}
      \item Limitadores
      \item Retificadores
      \item Dobradores de tensão
    \end{itemize}
  \item Transistores Bipolares
    \begin{itemize}
      \item Amplificadores e Buffers
      \item Fontes de tens'ao e corrente
      \item Portas lógicas (TTL)
    \end{itemize}

    \section*{Materiais para referência}
    O livro texto adotado para referências é a quarta edição de ``Microeletrônica'' por \emph{Sedra} e \emph{Smith} \cite{sedra1998microelectronic}, 
    no entanto, a fonte principal de consulta serão as notas de aulas. A escolha desta edição em específico se dá pelo enfoque em que é dado ao 
    transistor bipolar. 
    
 Para dúvidas a cerca de circuitos elétricos recomenda-se o livro de \emph{Orsini}\cite{orsini}.

 \section*{Avaliações}
 A média final será composta pela média simples de duas provas. Haverá uma prova substitutiva (substituindo a pior nota) ao final do curso,
 com o intuito de ajudar os alunos. A primeira prova terá duração aproximada de duas horas e trinta minutos, e as demais entre quatro e cinco horas.

 Testes surpresa poderão ocorrer no final das aulas, mas serão contabilizados apenas como bonificação nas nota finais.

 Calculadoras não serão sempre permitidas nas provas e caso forem, não haverá um aviso prévio. Na dúvida sempre leve sua calculadora (gráfica ou científica).

 A presença é obrigatória e haverá chamada.

 \section*{Recomendações}
 Sempre tire suas dúvidas durante ou no final de cada aula pois o ritmo da disciplina é intenso e a matéria é acumulativa. Não hesite em 
 procurar o professor fora da sala de aula, faz  parte de sua profissão sanar as dúvidas de seus alunos.
\end{itemize}
